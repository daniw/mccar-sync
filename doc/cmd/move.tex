\section{Move}

\subsection{Description}
Sends one Byte with Information for manual car steering. 
This Byte is divided in three parts. Those are steering, trimming and 
signals. \\
Steering uses Bits 0 to 3. This way each Bit represents a key if the car is 
controlled with the cursor keys of a keyboard. \\\\
\begin{tabular}{ll}
Steering Value  & Direction \\
0x00    & Stop \\
0x01    & Forward \\
0x02    & Turn right \\
0x03    & Forward with right curve \\
0x04    & Backward \\
0x05    & Unused (Stop) \\
0x06    & Backward with right corve \\
0x07    & Unused (Stop) \\
0x08    & Turn left \\
0x09    & Forward with left curve \\
0x0A    & Unused (Stop) \\
0x0B    & Unused (Stop) \\
0x0C    & Backward with left curve \\
0x0D    & Unused (Stop) \\
0x0E    & Unused (Stop) \\
0x0F    & Unused (Stop) \\
\end{tabular}\\\\
\begin{tabular}{ll}
Bit & Key on keyboard \\
0   & $\Uparrow$ \\
1   & $\Rightarrow$ \\
2   & $\Downarrow$ \\
3   & $\Leftarrow$ \\
\end{tabular}\\\\
Trimming uses Bits 4 and 5. Bit 4 trims to the left. Bit 5 trims to the 
right. \\
Bit 6 is used to control the signal horn. As long as it is set, the horn is 
active. 
Bit 7 is used to control the headlights. Every low to high transition of this 
Bit toggles the headlights. 
